% Options for packages loaded elsewhere
\PassOptionsToPackage{unicode}{hyperref}
\PassOptionsToPackage{hyphens}{url}
%
\documentclass[
]{book}
\usepackage{amsmath,amssymb}
\usepackage{lmodern}
\usepackage{iftex}
\ifPDFTeX
  \usepackage[T1]{fontenc}
  \usepackage[utf8]{inputenc}
  \usepackage{textcomp} % provide euro and other symbols
\else % if luatex or xetex
  \usepackage{unicode-math}
  \defaultfontfeatures{Scale=MatchLowercase}
  \defaultfontfeatures[\rmfamily]{Ligatures=TeX,Scale=1}
\fi
% Use upquote if available, for straight quotes in verbatim environments
\IfFileExists{upquote.sty}{\usepackage{upquote}}{}
\IfFileExists{microtype.sty}{% use microtype if available
  \usepackage[]{microtype}
  \UseMicrotypeSet[protrusion]{basicmath} % disable protrusion for tt fonts
}{}
\makeatletter
\@ifundefined{KOMAClassName}{% if non-KOMA class
  \IfFileExists{parskip.sty}{%
    \usepackage{parskip}
  }{% else
    \setlength{\parindent}{0pt}
    \setlength{\parskip}{6pt plus 2pt minus 1pt}}
}{% if KOMA class
  \KOMAoptions{parskip=half}}
\makeatother
\usepackage{xcolor}
\usepackage{longtable,booktabs,array}
\usepackage{calc} % for calculating minipage widths
% Correct order of tables after \paragraph or \subparagraph
\usepackage{etoolbox}
\makeatletter
\patchcmd\longtable{\par}{\if@noskipsec\mbox{}\fi\par}{}{}
\makeatother
% Allow footnotes in longtable head/foot
\IfFileExists{footnotehyper.sty}{\usepackage{footnotehyper}}{\usepackage{footnote}}
\makesavenoteenv{longtable}
\usepackage{graphicx}
\makeatletter
\def\maxwidth{\ifdim\Gin@nat@width>\linewidth\linewidth\else\Gin@nat@width\fi}
\def\maxheight{\ifdim\Gin@nat@height>\textheight\textheight\else\Gin@nat@height\fi}
\makeatother
% Scale images if necessary, so that they will not overflow the page
% margins by default, and it is still possible to overwrite the defaults
% using explicit options in \includegraphics[width, height, ...]{}
\setkeys{Gin}{width=\maxwidth,height=\maxheight,keepaspectratio}
% Set default figure placement to htbp
\makeatletter
\def\fps@figure{htbp}
\makeatother
\setlength{\emergencystretch}{3em} % prevent overfull lines
\providecommand{\tightlist}{%
  \setlength{\itemsep}{0pt}\setlength{\parskip}{0pt}}
\setcounter{secnumdepth}{5}
\usepackage{booktabs}
\ifLuaTeX
  \usepackage{selnolig}  % disable illegal ligatures
\fi
\usepackage[]{natbib}
\bibliographystyle{apalike}
\IfFileExists{bookmark.sty}{\usepackage{bookmark}}{\usepackage{hyperref}}
\IfFileExists{xurl.sty}{\usepackage{xurl}}{} % add URL line breaks if available
\urlstyle{same} % disable monospaced font for URLs
\hypersetup{
  pdftitle={Einführung in R},
  pdfauthor={Stephan Goerigk},
  hidelinks,
  pdfcreator={LaTeX via pandoc}}

\title{Einführung in R}
\author{Stephan Goerigk}
\date{2022-11-08}

\begin{document}
\maketitle

{
\setcounter{tocdepth}{1}
\tableofcontents
}
\hypertarget{uxfcber-dieses-skript}{%
\chapter*{Über dieses Skript}\label{uxfcber-dieses-skript}}
\addcontentsline{toc}{chapter}{Über dieses Skript}

Liebe Studierende,

dieses Skript soll Sie in die grundlegenden Analysewerkzeuge in R einführen, von der grundlegenden Kodierung und Analyse bis hin zur Datenverarbeitung, dem Plotten und der statistischen Inferenz.

Wenn R Ihre erste Programmiersprache ist, ist das völlig in Ordnung. Wir gehen alles Schritt für Schritt gemeinsam durch. Die Techniken in diesem Skript sind zwar auf die meisten Datenanalyseprobleme anwendbar, da wir jedoch aus der Psychologie kommen, werde ich den Kurs auf die Lösung von Analyseproblemen ausrichten, die in der psychologischen Forschung häufig auftreten.

Ich wünsche Ihnen Viel Erfolg und Spaß!

\hypertarget{warum-ist-r-so-gut}{%
\chapter{Warum ist R so gut?}\label{warum-ist-r-so-gut}}

\hypertarget{open-source}{%
\section{Open Source}\label{open-source}}

R ist zu 100 \% kostenlos und verfügt daher über eine große Unterstützergemeinschaft. Im Gegensatz zu SPSS, Matlab, Excel und JMP ist R völlig kostenlos und wird es auch immer bleiben. Das spart nicht nur Geld - es bedeutet auch, dass eine riesige Gemeinschaft von R-Programmierern ständig neue R-Funktionen und -Pakete in einer Geschwindigkeit entwickelt und verbreitet, die alle anderen Pakete in den Schatten stellt. Die Größe der R-Programmiergemeinschaft ist atemberaubend. Wenn Sie jemals eine Frage dazu haben, wie man etwas in R implementiert, wird eine schnelle Google-Suche Sie praktisch jedes Mal zur Antwort führen.

\hypertarget{vielseitigkeit}{%
\section{Vielseitigkeit}\label{vielseitigkeit}}

R ist unglaublich vielseitig. Sie können R für alles verwenden, von der Berechnung einfacher zusammenfassender Statistiken über die Durchführung komplexer Simulationen bis hin zur Erstellung großartiger Diagramme. Wenn Sie sich eine analytische Aufgabe vorstellen können, können Sie sie mit ziemlicher Sicherheit in R implementieren.

\hypertarget{r-markdown}{%
\section{R Markdown}\label{r-markdown}}

Mit RStudio, einem Programm, das Sie beim Schreiben von R-Code unterstützt, können Sie mit RMarkdown einfach und nahtlos R-Code, Analysen, Diagramme und geschriebenen Text zu eleganten Dokumenten an einem Ort kombinieren. Tatsächlich habe ich dieses gesamte Skript (Text, Formatierung, Diagramme, Code\ldots{} ja, alles) in RStudio mit R Markdown geschrieben Mit RStudio müssen Sie sich nicht mehr mit zwei oder drei Programmen herumschlagen, z. B. Excel, Word und SPSS, wo Sie die Hälfte Ihrer Zeit mit dem Kopieren, Einfügen und Formatieren von Daten, Bildern und Tests verbringen, sondern können alles an einem Ort erledigen, so dass nichts mehr falsch gelesen, getippt oder vergessen wird.

\hypertarget{transparenz}{%
\section{Transparenz}\label{transparenz}}

In R durchgeführte Analysen sind transparent, leicht weiterzugeben und reproduzierbar. Wenn Sie einen SPSS-Benutzer fragen, wie er eine bestimmte Analyse durchgeführt hat, wird er sich ggf. nicht daran erinnern, was er vor Monaten oder Jahren tatsächlich getan hat. Wenn Sie einen R-Anwender (der gute Programmiertechniken verwendet) fragen, wie er eine Analyse durchgeführt hat, sollte er Ihnen immer den genauen Code zeigen können, den er verwendet hat. Das bedeutet natürlich nicht, dass er die richtige Analyse verwendet oder sie korrekt interpretiert hat, aber mit dem gesamten Originalcode sollten etwaige Probleme völlig transparent sein! Dies ist eine Grundvoraussetzung für offene, replizierbare Forschung.

\hypertarget{r-materialien}{%
\chapter{R Materialien}\label{r-materialien}}

\hypertarget{cheat-sheets}{%
\section{Cheat Sheets}\label{cheat-sheets}}

In diesem Skript werden Sie viele neue Funktionen kennenlernen. Wäre es nicht schön, wenn jemand ein Wörterbuch mit vielen gängigen R-Funktionen erstellen würde? Ja, das wäre es, und zum Glück haben einige freundliche R-Programmierer genau das getan. Im Folgenden finden Sie eine Tabelle mit einigen der Funktionen, die ich empfehle. Ich empfehle Ihnen dringend, diese auszudrucken und die Funktionen zu markieren, wenn Sie sie lernen!

\href{https://cran.r-project.org/doc/contrib/Short-refcard.pdf}{Link zum Base R Cheat Sheet}

\href{https://www.rstudio.com/resources/cheatsheets/}{Link zu den R Studio Cheat Sheets}

Insbesondere die Cheat Sheets zu ggplot2 und dplyr kann ich Ihnen nur wärmstens ans Herz legen

\hypertarget{hilfe-und-inspiration-online}{%
\section{Hilfe und Inspiration online}\label{hilfe-und-inspiration-online}}

Eine Google Suche nach einem spezifischen R Problem bringt Sie (fast) immer an Ihr Ziel. Häufig findet man gute Antworten in den github Hilfsmatierialien einzelner Pakete, auf den Community Seiten von R Studio und in den Foren von stackoverflow.

\hypertarget{andere-buxfccher}{%
\section{Andere Bücher}\label{andere-buxfccher}}

Die Inhalte dieser Bücher sind nicht prüfunsrelevant.

Es gibt viele, viele ausgezeichnete Bücher über R. Hier sind zwei, die ich empfehlen kann (von denen eines sogar umsonst ist):

\href{https://www.amazon.com/Discovering-Statistics-Using-Andy-Field/dp/1446200469/ref=sr_1_2?ie=UTF8\&qid=1487759316\&sr=8-2\&keywords=statistics+with+r}{Discovering Statistics with R von Field, Miles and Field}

\href{http://r4ds.had.co.nz}{R for Data Science von Garrett Grolemund and Hadley Wickham}

\hypertarget{installation}{%
\chapter{Installation}\label{installation}}

Um R benutzen zu können müssen wir zwei Softwarepakete herunterladen:

\begin{itemize}
\tightlist
\item
  \textbf{R}
\item
  \textbf{RStudio}
\end{itemize}

R ist die Programmiersprache, mit der wir arbeiten. R-Studio ist eine Benutzeroberfläche, die uns das Programmieren mit R ungemein erleichtert.

\hypertarget{installation-von-r}{%
\section{Installation von R}\label{installation-von-r}}

Um R zu installieren, klicken Sie auf den, Ihrem Betriebssystem entsprechenden, Link und befolgen Sie die Anleitungen:

\begin{longtable}[]{@{}
  >{\raggedright\arraybackslash}p{(\columnwidth - 2\tabcolsep) * \real{0.5833}}
  >{\raggedright\arraybackslash}p{(\columnwidth - 2\tabcolsep) * \real{0.4167}}@{}}
\toprule()
\begin{minipage}[b]{\linewidth}\raggedright
Operating System
\end{minipage} & \begin{minipage}[b]{\linewidth}\raggedright
Link
\end{minipage} \\
\midrule()
\endhead
Windows & \url{http://cran.r-project.org/bin/windows/base/} \\
Mac & \url{http://cran.r-project.org/bin/macosx/} \\
\bottomrule()
\end{longtable}

Nach dieser Installation haben Sie bereits die volle Funktionalität des Programms. Sie werden jedoch feststellen, dass beinahe alle R-Nutzer RStudio zum programmieren nutzen, da dieses eine leichter nutzbare Oberfläche hat.
Tatsächlich müssen Sie nach der Installation von RStudio das R Basisprogramm nie wieder öffnen.

\hypertarget{installation-von-rstudio}{%
\section{Installation von RStudio}\label{installation-von-rstudio}}

Bitte installieren Sie dann RStudio - das Programm, über welches wir auf R zugreifen und mit dem wir unsere Skripte schreiben.

Um RStudio zu installieren, klicken Sie auf diesen Link und befolgen Sie die Anleitungen: \url{http://www.rstudio.com/products/rstudio/download/}

  \bibliography{book.bib,packages.bib}

\end{document}
